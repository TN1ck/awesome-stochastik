\documentclass[10pt,a4paper,parskip=half]{scrartcl}
\usepackage[utf8]{inputenc}
\usepackage{amsmath}
\usepackage{amsfonts}
\usepackage{amssymb}
\usepackage{mathpazo}
\usepackage{tikz}
\usetikzlibrary{patterns}
\usepackage[left=1cm, right=1cm,
top=1cm, bottom=1cm]{geometry}
\usepackage{fullpage}
\usepackage[german]{babel}
\usepackage{enumerate}
\setlength{\unitlength}{1cm}
\newcommand{\N}{\mathbb{N}}
\newcommand{\PP}{\mathbb{P}}
\newcommand{\A}{\mathcal{A}}
\newcommand{\R}{\mathbb{R}}
\parindent 0mm

\usepackage{color}
\usepackage{enumerate}



\begin{document}
\begin{center}
\textsc{\Large{Stochastik für Informatiker - Hausaufgabe 9}} \\
\end{center}
\begin{tabbing}
Tom Nick \hspace{1.4cm}\= 342225\\
Alexander Mühle\> 339497\\
Maximilian Bachl\> 341455
\end{tabbing}
\subsection*{Aufgabe 1}
Verteilung von X und Y
\begin{tabular}{c | c c  | c}
$X/Y$ & $4$ & $2$ & $\mathbb{P}(Y=y) $\\ \hline
2 & $\frac{1}{4}$ & $\frac{1}{6}$ & $\frac{1}{4} + \frac{1}{6}$ \\
5 & $\frac{1}{6}$ & $\frac{1}{4}$ & $\frac{1}{4} + \frac{1}{6}$ \\
3 & $\frac{1}{6}$ & $0$ & $\frac{1}{6}$ \\ \hline
$\mathbb{P}(X=x)$ & $\frac{1}{4} + \frac{2}{6}$ & $\frac{1}{4} + \frac{1}{6}$ & 1
\end{tabular}
\begin{enumerate}[(i)]
\item
$\mathbb{E}(X) = 2 \cdot (\frac{1}{4}+\frac{1}{6}) + 5 \cdot (\frac{1}{6} + \frac{1}{4}) + 3 \cdot \frac{1}{6} = \frac{163}{12}$ \\
$\mathbb{E}(X^2) = 4 \cdot (\frac{1}{4}+\frac{1}{6}) + 25 \cdot (\frac{1}{6} + \frac{1}{4}) + 9 \cdot \frac{1}{6} = \frac{41}{12}$ \\
$\mathbb{E}(Y) = 4 \cdot (\frac{1}{4}+\frac{1}{3}) + 2 \cdot (\frac{1}{6} + \frac{1}{4}) = \frac{19}{6}$ \\
$\mathbb{E}(Y^2) = 16 \cdot (\frac{1}{4}+\frac{1}{3}) + 4 \cdot (\frac{1}{6} + \frac{1}{4}) = 11$ \\
\item
$\mathbb{V}(X) = \mathbb{E}(X^2) - \mathbb{E}(X)^2 = \frac{163}{12} - \frac{1681}{144} = \frac{275}{144} \approx 1.91$ \\
$\mathbb{V}(Y) = \mathbb{E}(Y^2) - \mathbb{E}(Y)^2 = 11 - \frac{361}{36} = \frac{35}{36} \approx 0.97$ \\
\item
$Cov(X,Y) = \mathbb{E}[XY] - \mathbb{E}[X] \cdot \mathbb{E}[Y] = 8\cdot \frac 1 4 + 20 \cdot \frac 1 6 + 12 \cdot \frac 1 6 + 4 \cdot \frac 1 6 + 10 \cdot \frac 1 4 = 10\frac 1 2 - \frac{163}{12} \cdot \frac{19}{6} = -\frac{2341}{72} \approx -23$ \\
\item
$\rho (X,Y) = \frac{Cov(X,Y)}{\sqrt{\mathbb{V}(X)} \cdot \sqrt{\mathbb{V}(Y)}} = -\frac{23}{5 \cdot \sqrt{385}} \approx - 0.23 $ \\
\end{enumerate}


\subsection*{Aufgabe 2}
\begin{enumerate}[(i)]
\item
\end{enumerate}

\subsection*{Aufgabe 3}
\begin{enumerate}[(i)]
\item
$\mathbb{E}(X_k) = p_k$ \\
$\mathbb{E}(S_n) = \sum\limits_{k=1}^n \mathbb{E}(p_k) $

\item
% Nach Wikipedia Erwartungswert und Varianz gemacht.
$\mathbb{V}(S_n) = \sum\limits_{k=1}^n \mathbb V(X_k)$
Dies gilt, da die Variablen unabhängig und somit auch unkorelliert sind. 
$\sum\limits_{k=1}^n \mathbb V(X_k) = \sum\limits_{k=1}^n \mathbb E (X_k^2) - (\mathbb E (X_k))^2 = \sum\limits_{k=1}^n p_k - p_k^2$
\end{enumerate}

\subsection*{Aufgabe 4}
\begin{enumerate}[(i)]
\item
Abhängigkeit: \\
zu zeigen:
$\mathbb{P}(|X-Y|) \cdot \mathbb{P}(X+Y) = \mathbb{P}(|X-Y|) \cap \mathbb{P}(X+Y)$ \\
Gegenbeispiel: \\
Sei X = 1 und Y = 0
$\rightarrow \frac{1}{2} \cdot \frac{1}{2} \neq \frac{1}{2}$
\item
Korrelation: \\
$Cov(X+Y, |X-Y|) = \mathbb{E}((X+Y) \cdot |X-Y|) - \mathbb{E}(X+Y)\mathbb{E}(|X-Y|) = \frac{1}{2} - \frac{1}{2} = 0 \rightarrow$ unkorreliert \\
\end{enumerate}
\end{document}