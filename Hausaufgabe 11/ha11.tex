\documentclass[10pt,a4paper,parskip=half]{scrartcl}
\usepackage[utf8]{inputenc}
\usepackage{amsmath}
\usepackage{amsfonts}
\usepackage{amssymb}
\usepackage{mathpazo}
\usepackage{tikz}
\usetikzlibrary{patterns, automata, positioning}
\usepackage[left=1cm, right=1cm,
top=1cm, bottom=1cm]{geometry}
\usepackage{fullpage}
\usepackage[german]{babel}
\usepackage{enumerate}
\setlength{\unitlength}{1cm}
\newcommand{\N}{\mathbb{N}}
\newcommand{\PP}{\mathbb{P}}
\newcommand{\A}{\mathcal{A}}
\newcommand{\R}{\mathbb{R}}
\newcommand{\V}{\mathbb{V}}
\newcommand{\E}{\mathbb{E}}
\parindent 0mm

\usepackage{color}
\usepackage{enumerate}



\begin{document}
\begin{center}
\textsc{\Large{Stochastik für Informatiker - Hausaufgabe 11}} \\
\end{center}
\begin{tabbing}
Tom Nick \hspace{1.4cm}\= 342225\\
Alexander Mühle\> 339497\\
Maximilian Bachl\> 341455
\end{tabbing}

\section*{Aufgabe 1}
\begin{enumerate}
\item
$$\PP(\frac14 < X < \frac13) = F(\frac13) - F(\frac14) = \frac13 - \frac14 = \frac12$$
\begin{align*}
\PP(X < \frac12 \lor X > \frac23) &= \PP(X < \frac12) + \PP(\frac23 < X) \\
&= \PP(X < \frac12) + (\PP(X < 1) - \PP(X < \frac23) \\
&= F(\frac12) + (1 - F(\frac23)) \\
&= \frac12 + (1 - \frac23) = \frac12 + \frac13 = \frac56
\end{align*}
\item
$p \in (0,1)$ mit der Zufallsvariable $X$, wobei $\PP(X = 1) = p)$ und $\PP(X = 0) = 1 -p$. Somit ist $X$ \textit{fast} Bernoulli verteilt, womit gilt:
\begin{align*}
F_X(t) =
\begin{cases}
0 & \text{falls } t < 0 \\
1 - p &\text{falls } t \in [0,1] \\
1 &\text{falls } t \ge 1
\end{cases}
\end{align*}
\item
$n \in \N$, $X$ ist diskret verteilt auf der Menge $\{\frac1n, \frac2n, \dots ,1 \}$.
\begin{align*}
F_X(t) = \frac1n
\end{align*}
\end{enumerate}

\section*{Aufgabe 2}
Die Invariante der Uebergangsmatrix $P$ ist der Eigenvektor von $P$ zum Eigenwert $1$. Wir benutzen die \textit{Standardalgorhithmus} zum berechnen des Eigenwerts.
\begin{enumerate}
\item Da wir den Eigenvektor zum Eigenwert 1 berechen, muessen wir die Einheitsmatrix vom $P$ subtrahieren.
\begin{align*}
P = \left(\begin{matrix}
0.5 & 0.4 & 0.1 \\
0.3 & 0.4 & 0.3 \\
0.2 & 0.3 & 0.5
\end{matrix}\right) - I^3 =
\left(\begin{matrix}
-0.5 & 0.4 & 0.1 \\
0.3 & -0.6 & 0.3 \\
0.2 & 0.3 & -0.5
\end{matrix}\right)
\end{align*}
\item Die Matrix in ZSF bringen.
\begin{align*}
\left(\begin{matrix}
-0.5 & 0.4 & 0.1 \\
0.3 & -0.6 & 0.3 \\
0.2 & 0.3 & -0.5
\end{matrix}\right) \Rightarrow
\left(\begin{matrix}
1 & 0 & -1 \\
0 & 1 & -1 \\
0 & 0 & 0
\end{matrix}\right)
\end{align*}
Dies ergibt das LGS:
\begin{align*}
\pi(1) - \pi(3) = 0 \Leftrightarrow \pi(1) = \pi(3) \\
\pi(2) - \pi(3) = 0 \Leftrightarrow \pi(2) = \pi(3) \\
\pi(1) + \pi(2) + \pi(3) = 1 \\
\end{align*}
Daraus folgt:
$$\frac13 = \pi(1) = \pi(2) = \pi(3)$$
Somit ist die Invariante Verteilung $$\left(\frac13 ~ \frac13 ~ \frac13\right)$$
Probe:
$$\left(\frac13 ~ \frac13 ~ \frac13\right) \cdot \left(\begin{matrix}
0.5 & 0.4 & 0.1 \\
0.3 & 0.4 & 0.3 \\
0.2 & 0.3 & 0.5
\end{matrix}\right) = \left(\frac13 ~ \frac{11}{30} ~ \frac{3}{10}\right)$$
Da stimmt etwas nicht, ich habe die Rechnung per Hand und mit Wolfram-Alpha überprüft.
Es gibt also keine Invariante :(.
\end{enumerate}

\section*{Aufgabe 3}

\begin{align*}
\begin{pmatrix} p_{1,1} & \dots & p_{1,N} \\ \vdots & \ddots & \vdots \\ p_{N,1} & \dots & p_{N,N} \end{pmatrix}\begin{pmatrix} \frac 1 N \\ \vdots \\ \frac 1 N \end{pmatrix} &= \begin{pmatrix} \sum_{j=1}^{N} p_{1,j} \cdot \frac 1 N \\ \vdots \\ \sum_{j=1}^{N} p_{N,j} \cdot \frac 1 N \end{pmatrix} \\
&= \begin{pmatrix} \frac 1 N \sum_{j=1}^{N} p_{1,j} \\ \vdots \\ \frac 1 N \sum_{j=1}^{N} p_{N,j} \end{pmatrix}\\
&= \begin{pmatrix} \frac 1 N \cdot 1 \\ \vdots \\ \frac 1 N \cdot 1 \end{pmatrix}
\\
&= \begin{pmatrix} \frac 1 N \\ \vdots \\ \frac 1 N \end{pmatrix}
\end{align*}
\section*{Aufgabe 4}
\begin{enumerate}
\item
\begin{center}
    \begin{tikzpicture}[auto,bend angle=30,node distance=3cm]
      % Zustaende
      \node[state, initial text=,initial where=left, accepting]  (q0)  {$q_0$};
      \node[state]                                                (q1) [above left of=q0] {$q_1$};
      \node[state]                                                (q2) [left of=q1] {$q_2$};
      \node[state]                                                (q3) [below left of=q0] {$q_3$};
      \node[state]                                     (q4) [below right  of=q0] {$q_4$};
      % Pacman Kugeln
      % Transitionen    
      \path[->, swap] (q0) edge node {$\mathsf{a,b}$} (q1);
      \path[->, bend right, swap] (q1) edge node {$\mathsf{a,b}$} (q2);
      \path[->, bend right, swap] (q2) edge node {$\mathsf{c}$} (q3);
      \path[->, bend right, swap] (q3) edge node {$\mathsf{a,b}$} (q4);
      \path[->, swap] (q4) edge node {$\mathsf{a,b}$} (q0);
      \path[->, loop right] (q1) edge node {$\mathsf{a}$} (q1);
    \end{tikzpicture}
  \end{center}
$$P = \left(\begin{matrix}
0 & \frac12 & 0 & 0 & \frac12 & \\
\frac12 & 0 & \frac12 & 0 & 0 & \\
0 & \frac12 & 0 & \frac12 & 0 & \\
0 & 0 & \frac12 & 0 & \frac12 & \\
\frac12 & 0 & 0 & \frac12 & 0
\end{matrix}\right)$$
\end{enumerate}


\end{document}









