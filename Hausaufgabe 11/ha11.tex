\documentclass[10pt,a4paper,parskip=half]{scrartcl}
\usepackage[utf8]{inputenc}
\usepackage{amsmath}
\usepackage{amsfonts}
\usepackage{amssymb}
\usepackage{mathpazo}
\usepackage{tikz}
\usetikzlibrary{patterns, automata, positioning}
\usepackage[left=1cm, right=1cm,
top=1cm, bottom=1cm]{geometry}
\usepackage{fullpage}
\usepackage[german]{babel}
\usepackage{enumerate}
\setlength{\unitlength}{1cm}
\newcommand{\N}{\mathbb{N}}
\newcommand{\PP}{\mathbb{P}}
\newcommand{\A}{\mathcal{A}}
\newcommand{\R}{\mathbb{R}}
\newcommand{\V}{\mathbb{V}}
\newcommand{\E}{\mathbb{E}}
\parindent 0mm

\usepackage{color}
\usepackage{enumerate}



\begin{document}
\begin{center}
\textsc{\Large{Stochastik für Informatiker - Hausaufgabe 11}} \\
\end{center}
\begin{tabbing}
Tom Nick \hspace{1.4cm}\= 342225\\
Alexander Mühle\> 339497\\
Maximilian Bachl\> 341455
\end{tabbing}

\section*{Aufgabe 1}
\begin{enumerate}
\item
$$\PP(\frac14 < X < \frac13) = F(\frac13) - F(\frac14) = \frac13 - \frac14 = \frac12$$
\begin{align*}
\PP(X < \frac12 \lor X > \frac23) &= \PP(X < \frac12) + \PP(\frac23 < X) \\
&= \PP(X < \frac12) + (\PP(X < 1) - \PP(X < \frac23) \\
&= F(\frac12) + (1 - F(\frac23)) \\ 
&= \frac12 + (1 - \frac23) = \frac12 + \frac13 = \frac56
\end{align*}

\section*{Aufgabe 3}

\begin{align*}
\begin{pmatrix} p_{1,1} & \dots & p_{1,N} \\ \vdots & \ddots & \vdots \\ p_{N,1} & \dots & p_{N,N} \end{pmatrix}\begin{pmatrix} \frac 1 N \\ \vdots \\ \frac 1 N \end{pmatrix} &= \begin{pmatrix} \sum_{j=1}^{N} p_{1,j} \cdot \frac 1 N \\ \vdots \\ \sum_{j=1}^{N} p_{N,j} \cdot \frac 1 N \end{pmatrix} \\
&= \begin{pmatrix} \frac 1 N \sum_{j=1}^{N} p_{1,j} \\ \vdots \\ \frac 1 N \sum_{j=1}^{N} p_{N,j} \end{pmatrix}\\
&= \begin{pmatrix} \frac 1 N \cdot 1 \\ \vdots \\ \frac 1 N \cdot 1 \end{pmatrix}
\\
&= \begin{pmatrix} \frac 1 N \\ \vdots \\ \frac 1 N \end{pmatrix}
\end{align*}

\end{enumerate}

\end{document}









