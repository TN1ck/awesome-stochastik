\documentclass[10pt,a4paper,parskip=half]{scrartcl}
\usepackage[utf8]{inputenc}
\usepackage{amsmath}
\usepackage{amsfonts}
\usepackage{amssymb}
\usepackage{mathpazo}
\usepackage{tikz}
\usetikzlibrary{patterns, automata, positioning}
\usepackage[left=1cm, right=1cm,
top=1cm, bottom=1cm]{geometry}
\usepackage{fullpage}
\usepackage[german]{babel}
\usepackage{enumerate}
\setlength{\unitlength}{1cm}
\newcommand{\N}{\mathbb{N}}
\newcommand{\PP}{\mathbb{P}}
\newcommand{\A}{\mathcal{A}}
\newcommand{\R}{\mathbb{R}}
\newcommand{\V}{\mathbb{V}}
\newcommand{\E}{\mathbb{E}}
\parindent 0mm

\usepackage{color}
\usepackage{enumerate}



\begin{document}
\begin{center}
\textsc{\Large{Stochastik für Informatiker - Hausaufgabe 10}} \\
\end{center}
\begin{tabbing}
Tom Nick \hspace{1.4cm}\= 342225\\
Alexander Mühle\> 339497\\
Maximilian Bachl\> 341455
\end{tabbing}

\section*{Aufgabe 1}
\begin{enumerate}[a)]
\item\leavevmode\vadjust{\vspace{-\baselineskip}}\newline
\begin{tikzpicture}[shorten >=1pt,node distance=2cm,on grid,auto] 
   \node[state] at (2, 0) (1) {$1$}; 
   \node[state] at (0, 0) (2) {$2$}; 
   \node[state] at (4, 0) (3) {$3$}; 
   \node[state] at (2, -2) (4) {$4$};
   \path[->]
    (1) edge [bend left] node {0.5} (2)
        edge [bend left]node {0.25} (3)
        edge  [bend left]node {0.25} (4)
    (2) edge [bend left] node {1} (1)
    (3) edge [bend left] node {1} (1) 
    (4) edge [bend left] node {1} (1); 
\end{tikzpicture}
\item
Länge 2:

1 2 1\\
1 3 1\\
1 4 1\\
2 1 2\\
2 1 3\\
2 1 4\\
3 1 3\\
3 1 2\\
3 1 4\\
4 1 4\\
4 1 2\\
4 1 3\\

Länge 3:

1 2 1 2\\
1 2 1 3\\
1 2 1 4\\
1 3 1 2\\
1 3 1 3\\
1 3 1 4\\
1 4 1 2\\
1 4 1 3\\
1 4 1 4\\
2 1 2 1\\
2 1 3 1\\
2 1 4 1\\
3 1 3 1\\
3 1 2 1\\
3 1 4 1\\
4 1 4 1\\
4 1 2 1\\
4 1 3 1\\
\item
$\begin{pmatrix} a & b & c & d \end{pmatrix}\begin{pmatrix} 0 & 0.5 & 0.25 & 0.25 \\ 1 & 0 & 0 & 0\\ 1 & 0 & 0 & 0\\ 1 & 0 & 0 & 0 \end{pmatrix} = \begin{pmatrix} b + c + d & \frac a 2 & \frac a 4 & \frac a 4 \end{pmatrix}$

Daraus folgt dieses Gleichungssystem:

$a = b + c + d$\\
$b = \frac a 2$\\
$c = \frac a 4$\\
$d = \frac a 4$\\

Eine Lösung ist: $a = \frac 1 2$, $b = \frac 1 4$, $c = \frac 1 8$, $d = \frac 1 8$.
\end{enumerate}
\section*{Aufgabe 2}
\begin{enumerate}[a)]
\item
Nein, es ist nicht irreduzibel, da man aus den Zuständen 1 und 2 nicht nach 3 oder 4 kommen kann und umgekehrt.
\item
$M = \{(a~ b~ c~ d)\mid a = b \land c = d \land a + b + c + d = 1\}$
\end{enumerate}
\section*{Aufgabe 3}
Wir substituieren folgendermaßen:

\begin{align*}
k\alpha + (1-k)(1-\beta) - \frac{1 -\beta}{2 -\alpha-\beta} &= k\alpha + 1 -\beta -k +k\beta - \frac{1 -\beta}{2 -\alpha-\beta} \\
&= k(\alpha + \beta - 1) + 1 - \beta - \frac{1 -\beta}{2 -\alpha-\beta} \\
&= k(\alpha + \beta - 1) + \frac{(1 -\beta)(2 -\alpha-\beta)}{2 -\alpha-\beta} - \frac{1 -\beta}{2 -\alpha-\beta} \\
&= k(\alpha + \beta - 1) + \frac{(1 -\beta)(1 -\alpha-\beta)}{2 -\alpha-\beta} \\
&= k(\alpha + \beta - 1) - \frac{(1 -\beta)(-1 +\alpha+\beta)}{2 -\alpha-\beta} \\
&= (\alpha+\beta-1)(k-\frac{1-\beta}{2-\alpha-\beta})
\end{align*}

Nun für den zweiten Term:
Wir benutzen Induktion:

Induktionsanfang: 
\begin{align*}
P(X_1 = 1) = \frac{1-\beta}{2-\alpha-\beta} + (\alpha +\beta -1)^1\left(P(X_0 = 1) - \frac {1-\beta}{2-\alpha-\beta}\right)
\end{align*}
Dies gilt aufgrund des obigen Beweises.

Induktionsvoraussetzung:
\begin{align*}
P(X_{n} = 1) = \frac{1-\beta}{2-\alpha-\beta} + (\alpha +\beta -1)^n\left(P(X_0 = 1) - \frac {1-\beta}{2-\alpha-\beta}\right)
\end{align*}

Induktionsschritt:
\begin{align*}
P(X_{n+1} = 1) &= \frac{1-\beta}{2-\alpha-\beta} + (\alpha +\beta -1)\left(P(X_n = 1) - \frac {1-\beta}{2-\alpha-\beta}\right)\\
&\overset{\text{IV}}{=} \frac{1-\beta}{2-\alpha-\beta} + (\alpha +\beta -1)\left(\frac{1-\beta}{2-\alpha-\beta} + (\alpha +\beta -1)^n\left(P(X_0 = 1) - \frac {1-\beta}{2-\alpha-\beta}\right) - \frac {1-\beta}{2-\alpha-\beta}\right)\\
&= \frac{1-\beta}{2-\alpha-\beta} + (\alpha +\beta -1)\frac{1-\beta}{2-\alpha-\beta} + (\alpha +\beta -1)^{n+1}\left(P(X_0 = 1) - \frac {1-\beta}{2-\alpha-\beta}\right)\\
&-(\alpha +\beta -1)\frac{1-\beta}{2-\alpha-\beta}\\
&= \frac{1-\beta}{2-\alpha-\beta} + (\alpha +\beta -1)^{n+1}\left(P(X_0 = 1) - \frac {1-\beta}{2-\alpha-\beta}\right)
\end{align*}

\section*{Aufgabe 4}
\begin{enumerate}[a)]
\item Diese Markov-Kette ist irreduzibel, da man von jedem Zustand mit pos.~Wahrscheinlichkeit in jeden anderen gelangen kann.

Die Kette ist nicht aperiodisch, da der Weg von jedem Knoten zu sich selbst immer als Länge ein vielfaches von 5 hat.
\item
Diese Kette hat nur noch mehr Übergänge als die in a). Also ist sie erst recht irreduzibel.

Sie ist aperiodisch, da der ggT von 5 und 3 sowie von 5 und 4 eins ist. Das ist die Länge des Weges, je nachdem, von welchem Knoten man ausgeht.
\item Von 4 kann ich niemals die 1 erreichen. Also ist die Kette nicht irreduzibel und somit auch nicht aperiodisch.
\end{enumerate}
\end{document}









