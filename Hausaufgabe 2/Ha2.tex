\documentclass[10pt,a4paper,parskip=half]{scrartcl}
\usepackage[utf8]{inputenc}
\usepackage{amsmath}
\usepackage{amsfonts}
\usepackage{amssymb}
\usepackage{mathpazo}
\usepackage{tikz}
\usetikzlibrary{patterns}
\usepackage{stmaryrd} % Für den Widerspruchsblitz :D
\usepackage[left=1cm, right=1cm,
top=1cm, bottom=1cm]{geometry}
\usepackage{fullpage}
\usepackage[german]{babel}
\usepackage{enumerate}
\setlength{\unitlength}{1cm}
\newcommand{\N}{\mathbb{N}}
\newcommand{\PP}{\mathbb{P}}
\newcommand{\A}{\mathcal{A}}
\newcommand{\R}{\mathbb{R}}
\parindent 0mm

\usepackage{color}
\usepackage{enumerate}



\begin{document}
\begin{center}
\textsc{\Large{Stochastik für Informatiker - Hausaufgabe 2}} \\
\end{center}
\begin{tabbing}
Tom Nick \hspace{1.4cm}\= 342225\\
Alexander Mühle\> 339497\\
Maximilian Bachl\> 341455
\end{tabbing}
\section*{1. Aufgabe}
\begin{enumerate}[(i)]
\item 
	Es gibt 26 verschiedene Buchstaben und 10 Ziffern. Bei dieser Aufgabe ist die Reihenfolge entscheidend und man kann alle Buchstaben und Ziffern beliebig oft wiederverwenden, also arbeiten wir mit Zurücklegen.

	Somit kommt man auf $25^2 \cdot 10^5 = 67600000$ Möglichkeiten.
\item
	Nun arbeiten wir ohne Zurücklegen, also kommen wir auf $26 \cdot 25 \cdot \frac{10!}{5!} = 19656000$ Möglichkeiten.
\end{enumerate}
\section*{2. Aufgabe}
\begin{enumerate}[(i)]
\item
	Sind beide Bücher zum gleichen Thema, so kommt man auf 
	\begin{align*}
	\begin{pmatrix}6\\2\end{pmatrix}+\begin{pmatrix}7\\2\end{pmatrix}+\begin{pmatrix}4\\2\end{pmatrix} = 42
	\end{align*}
	Die Reihenfolge ist hier nämlich egal. Wir addieren außerdem die Möglichkeiten. Die erste Zahl ist für die Mathematik, die zweite für Chemie und die dritte für Ökonomie.
\item
	Offensichtlich gibt es beim Auswählen von zwei Büchern aus der Menge genau 2 Fälle: Entweder man wählt zwei gleiche Bücher aus, oder zwei verschiedene, andere Möglichkeiten gibt es nicht. Somit ist das Ergebnis hier die Zahl aller Möglichkeiten minus der Zahl der gleichen aus der vorigen Aufgabe.
	\begin{align*}
	\begin{pmatrix}17\\2\end{pmatrix} - 42 = 136 - 42 = 94
	\end{align*}
\end{enumerate}
\section*{3. Aufgabe}
\begin{enumerate}[(i)]
\item
	Wir wählen zuerst aus den 15 eine Gruppe von 7 aus. Aus den verbleibenden 8 wählen wir 5 aus. Nun bleiben noch 3. Hier gibt es nur eine Möglichkeit eine 3er-Gruppe aus ihnen zu bilden.
	\begin{align*}
	\begin{pmatrix}15\\7\end{pmatrix}\cdot\begin{pmatrix}8\\5\end{pmatrix}\cdot\begin{pmatrix}3\\3\end{pmatrix} = 360360
	\end{align*}
\item
	Für die Marschordnung (hier spielen die Gruppen keine Rolle) gibt es $15! = 1307674368000$ verschiede Möglichkeiten.
\end{enumerate}
\section*{4. Aufgabe}
\begin{enumerate}[(i)]
\item
	Hier gehen wir analog zur Tutoriumsaufgabe vor. Es können aber auch Schüler leer ausgehen. Also kommt man auf das folgende Ergebnis:
	\begin{align*}
	&\begin{pmatrix}n+k-1\\k-1\end{pmatrix}=\begin{pmatrix}8+4-1\\4-1\end{pmatrix}\\
	&= \begin{pmatrix}11\\3\end{pmatrix} = 165
	\end{align*}
\item
	Ebenfalls analog zur Tutoriumsaufgabe kommt man hier auf die folgende Anzahl an Möglichkeiten:
	\begin{align*}
	&\begin{pmatrix}n-1\\k-1\end{pmatrix}=\begin{pmatrix}8-1\\4-1\end{pmatrix}\\
	&= \begin{pmatrix}9\\3\end{pmatrix} = 84
	\end{align*}
\end{enumerate}

\end{document}

