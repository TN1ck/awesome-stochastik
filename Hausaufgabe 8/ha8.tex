\documentclass[10pt,a4paper,parskip=half]{scrartcl}
\usepackage[utf8]{inputenc}
\usepackage{amsmath}
\usepackage{amsfonts}
\usepackage{amssymb}
\usepackage{mathpazo}
\usepackage{booktabs}
\usepackage{tikz}
\usetikzlibrary{patterns}
\usepackage[left=1cm, right=1cm,
top=1cm, bottom=1cm]{geometry}
\usepackage{fullpage}
\usepackage[german]{babel}
\usepackage{enumerate}
\setlength{\unitlength}{1cm}
\newcommand{\N}{\mathbb{N}}
\newcommand{\PP}{\mathbb{P}}
\newcommand{\E}{\mathbb{E}}
\newcommand{\A}{\mathcal{A}}
\newcommand{\R}{\mathbb{R}}
\parindent 0mm

\usepackage{color}
\usepackage{enumerate}
\renewcommand*\arraystretch{1.5}


\begin{document}
\begin{center}
\textsc{\Large{Stochastik für Informatiker - Hausaufgabe 8}} \\
\end{center}
\begin{tabbing}
Tom Nick \hspace{1.4cm}\= 342225\\
Alexander Mühle\> 339497\\
Maximilian Bachl\> 341455
\end{tabbing}
\section*{Aufgabe 1}
\begin{enumerate}
\item \begin{align*}
\Omega = \{(Z), (K, Z), (K, K, Z) , (K, K, K)\} \\
\end{align*}
Es wird solange geworfen, bis das erste mal Zahl oder $3x$ Kopf kommt, also ist $\PP$ definiert als:
\begin{align*}
\PP(\{(Z)\}) &= \frac{1}{2} \\
\PP(\{(K, Z)\} &= \frac{1}{4} \\
\PP(\{(K, K, Z)\} &= \frac{1}{8} \\
\PP(\{K,K,K)\} &= \frac{1}{8} \\
\end{align*}
Also ist der Erwartungswert für die benötigten Züge $X$:
\begin{align*}
\E(X) &= 1 \cdot \PP(X = 1) + 2\cdot\PP(X=2)+3\cdot\PP(X=3) \\
\E(X) &= 1 \cdot \frac{1}{2} + 2\cdot \frac{1}{4} + 3 \cdot 2 \cdot \frac{1}{8} \\
\E(X) &= 1 \frac{3}{4}
\end{align*}
\item U bezeichnet die Augenzahl.
$$\E(\frac{1}{U}) = \sum_{i=1}^{6} = \frac{1}{i} \cdot \frac{1}{6} = \frac{2}{7} $$
\end{enumerate}
\section*{Aufgabe 2}
\begin{enumerate}
\item $X$ ist eine zum Parameter $\lambda$ Poisson-verteilte Zufallsvariable, $U \in \R$.
\begin{align*}
\E(e^{UX}) &= \sum^{\infty}_{k=0} e^{Uk}\frac{\lambda^k}{k!} e^{-\lambda} \\
&= e^{-\lambda}\cdot \sum_{k=0}^{\infty} \frac{(e^u\lambda)^k}{k!} = e^{e^u\lambda - \lambda}
\end{align*}
\item Sei $X$ eine zum Parameter $\lambda$ Poisson-verteilte Zufallsvariable.
\begin{align*}
\E\left(\frac{1}{1+X}\right) &= \sum^{\infty}_{k=0} \frac{1}{1+k}\frac{\lambda^k}{k!} e^{-\lambda} \\
&= \sum^{\infty}_{k=0} \frac{\lambda^k}{(k+1)!} e^{-\lambda} \\
&= \sum^{\infty}_{k=1} \frac{\lambda^{k}}{k!}\cdot \frac1\lambda e^{-\lambda} \\
&= (e^\lambda - 1) \cdot \frac1\lambda e^{-\lambda} \\
&= (\frac1\lambda  - \frac1\lambda e^{-\lambda}) \\
&= \frac{1}{\lambda}(1 - e^{-\lambda}) \\
\end{align*}
\end{enumerate}
\section*{Aufgabe 3}
Um einen fairen Spielbetrag berechnen zu können, müssen wir zuerst die erwarteten Gewinne/Verluste berechnen. Sei X der zu erwartene Gewinn der Spielerin.
\begin{align*}
\E(X) &= 2 \cdot \frac{1}{6} + 6 \cdot \frac{1}{6} + 10 \cdot \frac{1}{6}  - 2 \cdot \frac{1}{6} + 4 \cdot \frac{1}{6} + 6 \frac{1}{6} \\
&= 3 - 2 = 1
\end{align*}
Also müsste der Einsatz des Spiels 1 euro betragen, damit es fair für beide Parteien ist.
\section*{Aufgabe 4}
Sei $n_1$ die Anzahl der Sprünge nach links und $n_2$ die nach rechts. Es gilt $n_1 + n_2 = N$
Mit der Wahrscheinlichkeit $p$ springt sie nach rechts und mit $p - 1 = q$ nach links.

Es gibt $\binom{N}{n_1}$ Möglichkeiten um exakt $n_1$ Schritte nach links und $\binom{N}{n_2}$ Möglichkeiten um $n_2$ nach rechts zu springen.
Somit beträgt die Wahrscheinlichkeit um eine bestimmte geordnete Sequenz von $n_1$ und $n_2$ Sprüngen ist demnach $q^{n_1}$ bzw. $p^{n_2}$.

Dies ist also eine Binomialverteilung, somit beträgt der Erwartungswert der Position der Biene noch $N$ Sprüngen:
$$\E(X_n) = n_2 \cdot p - n_1 \cdot q$$
\end{document}




