\documentclass[10pt,a4paper,parskip=half]{scrartcl}
\usepackage[utf8]{inputenc}
\usepackage{amsmath}
\usepackage{amsfonts}
\usepackage{amssymb}
\usepackage{mathpazo}
\usepackage{tikz}
\usetikzlibrary{patterns}
\usepackage[left=1cm, right=1cm,
top=1cm, bottom=1cm]{geometry}
\usepackage{fullpage}
\usepackage[german]{babel}
\usepackage{enumerate}
\setlength{\unitlength}{1cm}
\newcommand{\N}{\mathbb{N}}
\newcommand{\PP}{\mathbb{P}}
\newcommand{\A}{\mathcal{A}}
\newcommand{\R}{\mathbb{R}}
\parindent 0mm

\usepackage{color}
\usepackage{enumerate}



\begin{document}
\begin{center}
\textsc{\Large{Stochastik für Informatiker - Hausaufgabe 5}} \\
\end{center}
\begin{tabbing}
Tom Nick \hspace{1.4cm}\= 342225\\
Alexander Mühle\> 339497\\
Maximilian Bachl\> 341455
\end{tabbing}
\subsection*{Aufgabe 1}

\begin{enumerate}[(i)]
\item $\mathbb P (X=1) = \mathbb P(\{\omega_1\}) = \frac 1 2$

$\mathbb P (Y=1) = \mathbb P(\{\omega_1, \omega_3\}) = 2\frac 1 4 = \frac 1 2$

$\mathbb P (X=2) = \mathbb P(\{\omega_2, \omega_3\}) = 2\frac 1 4 = \frac 1 2$

$\mathbb P (Y=2) = \mathbb P(\{\omega_1\}) = \frac 1 2$

Somit sieht man, dass die Zufallsvariablen gleich verteilt sind.

\item\leavevmode\vadjust{\vspace{-\baselineskip}}\newline
\begin{tabular}{c | c c c}
& $w_1$ & $w_2$ & $w_3$ \\ \hline
X+Y & 3 & 3 & 3 \\
XYZ & 2 & 4 & 2 \\
$X^Y$ & 1 & 2 & 2 \\
X & 1 &2 &2 \\
Y & 2 &1 &1 \\
Z &1 &2 &1
\end{tabular}

Nun kann man die Einzelwahrscheinlichkeiten einfach ablesen:

$\mathbb P (X+Y = 3) = 1$

$\mathbb P (XYZ = 2) = \frac 3 4$\\
$\mathbb P (XYZ = 4) = \frac 1 4$

$\mathbb P (X^Y = 1) = \frac 1 2$\\
$\mathbb P (X^Y = 2) = \frac 1 2$

\end{enumerate}

\subsection*{Aufgabe 2}
Buchstabe ist falsch: $p$\\
Buchstabe ist richtig: $1-p$

$n = $ Anzahl der Buchstaben im Buch\\
$X = $ Anzahl falsch gedruckter Buchstaben

Es handelt sich hier um eine Binomialverteilung: \\
$\mathbb P (X = k) = {3 \choose 2} \cdot p^k \cdot (1-p)^k$

\subsection*{Aufgabe 3}
Es gibt folgende Möglichkeiten:

\begin{itemize}
\item $R=n$

$\mathbb P (\{rrrr\}) = 1 \cdot\frac 1 {n+1} \cdot 1 \cdot \frac 1 {n+1}$\\
$\mathbb P (\{rbbr\}) = 1 \cdot\frac n {n+1} \cdot \frac {1} n \cdot \frac 1 {n+1}$\\
\item $R=n-1$

$\mathbb P (\{rrrb\}) = 1 \cdot\frac 1 {n+1} \cdot 1 \cdot \frac n {n+1}$\\
$\mathbb P (\{rbbb\}) = 1 \cdot\frac n {n+1} \cdot \frac {n-1} n \cdot \frac 2 {n+1}$\\
$\mathbb P (\{rbrr\}) = 1 \cdot\frac n {n+1} \cdot \frac {n-1} n \cdot \frac 2 {n+1}$\\
\item $R=n-2$

$\mathbb P (\{rbrb\}) = 1 \cdot\frac n {n+1} \cdot \frac {n-1} n \cdot \frac {n-1} {n+1}$\\
\end{itemize}

Somit gilt:

$\mathbf P(R=n) = \mathbb P (\{rrrr\}) + \mathbb P (\{rbbr\})$\\
$\mathbf P(R=n-1) = \mathbb P (\{rrrb\}) + \mathbb P (\{rbbb\}) + \mathbb P (\{rbrr\})$\\
$\mathbf P(R=n-2) = \mathbb P (\{rbrb\})$

\subsection*{Aufgabe 4}
\begin{enumerate}[(i)]
\item
$\mathbf P(X = n) = p(n)$
\item
$\mathbf P(X = n) = \frac 1 {n+1}$

Für $\mathbf P(X>n) = p'(n)$, wobei wir für $p'$ folgende rekursive Definition angeben:\\
$p'(1) = p(1)$\\
$p'(n) = p'(n-1) \cdot p(n)$

Beispielsweise sind dies die ersten Funktionswerte von $p$ und $p'$:\\
$p(1) = 0.5$\\
$p'(1) = 0.5$\\
$p(2) = 0.\overline3$\\
$p'(2) = 0.1\overline6$


\end{enumerate}
\end{document}
