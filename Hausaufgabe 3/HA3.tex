\documentclass[10pt,a4paper,parskip=half]{scrartcl}
\usepackage[utf8]{inputenc}
\usepackage{amsmath}
\usepackage{amsfonts}
\usepackage{amssymb}
\usepackage{mathpazo}
\usepackage{tikz}
\usetikzlibrary{patterns}
\usepackage[left=1cm, right=1cm,
top=1cm, bottom=1cm]{geometry}
\usepackage{fullpage}
\usepackage[german]{babel}
\usepackage{enumerate}
\setlength{\unitlength}{1cm}
\newcommand{\N}{\mathbb{N}}
\newcommand{\PP}{\mathbb{P}}
\newcommand{\A}{\mathcal{A}}
\newcommand{\R}{\mathbb{R}}
\parindent 0mm

\usepackage{color}
\usepackage{enumerate}



\begin{document}
\begin{center}
\textsc{\Large{Stochastik für Informatiker - Hausaufgabe 3}} \\
\end{center}
\begin{tabbing}
Tom Nick \hspace{1.4cm}\= 342225\\
Alexander Mühle\> 339497\\
Maximilian Bachl\> 341455
\end{tabbing}
\section*{1. Aufgabe}
Eine Studentin im Studiengang A muss 4 Klausuren schreiben und hat eine ausreichende
Note, wenn sie mindestens 2 davon besteht. Ein Student aus Studiengang B muss mindestens eine von drei Klausuren bestehen. Fur wen ist es einfacher eine ausreichende Note zu bekommen, wenn die Wahrscheinlichkeit jede einzelne Klausur zu bestehen gleich ist und die Klausuren unabhangig voneinander bestanden werden?
\begin{center}
$\Omega = (k_1, k_2, k_3, k_4) | k_i \in (B, \mathbb{N}) $
\item
\item[-]
\begin{description}
Ist eine Binomialverteilung
\end{description}
\item[-]
\begin{description}
Versuche sind unabhängig
\end{description}
\item[-]
\begin{description}
Ergebnisse sind binär
\end{description}
\begin{center} 
$\Rightarrow \textbf{Studentin A } \boxed{\mathbb{P} (k) = \left( \begin{array}{cc} n \\ k \end{array} \right) p^k (1- p)^{n-k}} \text{ mit n = 4 und k = 2}$
\end{center}
\item[$\Rightarrow$]
\begin{description}
$\sum\limits_{i \in {\{1,2,3,4\}}}^n \mathbb{P} (i) = 1 - \left( (1-p)^n + 4p(1-p)^{n-1} \right)$
\end{description}
\begin{center} 
$\Rightarrow \textbf{Studentin B } \boxed{\mathbb{P} (k) = \left( \begin{array}{cc} n \\ k \end{array} \right) p^k (1- p)^{n-k}} \text{ mit n = 3 und k = 1}$
\end{center}
\item[$\Rightarrow$]
\begin{description}
$\sum\limits_{i \in {\{1,2,3\}}}^n \mathbb{P} (i) = 1 - \left( (1-p)^n \right)$
\end{description}
\begin{center}
$1 - \left( (1-p)^4 + 4p(1-p)^{3} \right) < 1 - \left( (1-p)^3 \right) $\\[1cm]
\end{center}
Es ist einfacher für die Studentin A eine ausreichende Note zu bekommen, da ihre Wahrscheinlichkeit eine zu bestehen höher ist, als die Wahrscheinlichkeit des Studenten B zwei von vier zu bestehen.
\end{center}

\section*{2. Aufgabe}
Wir betrachten das Experiment, das aus dem Werfen zweier fairer Würfel besteht.
\begin{flushleft}
\item[i)]
\begin{description}
Bedingte Wahrscheinlichkeit
\end{description}
\end{flushleft}
\begin{displaymath}
  \frac{ \mathbb{P} \{(4,5),(6,5),(5,5),(5,6),(5,4)\}}{\mathbb{P} \{(1,5),(2,5),(3,5),(4,5),(5,5),(5,6),(6,5),(5,4),(5,3),(5,2),(5,1)\}} = \frac{5}{9}
\end{displaymath}
\begin{flushleft}
\item[ii)]
\begin{description}
Bedingte Wahrscheinlichkeit
\end{description}
\end{flushleft}
\begin{displaymath}
	\frac{ \mathbb{P} \{(4,5),(6,5),(5,6),(5,4)\}}{\mathbb{P} \{(1,5),(2,5),(3,5),(4,5),(5,6),(6,5),(5,4),(5,3),(5,2),(5,1)\}} = \frac{4}{8} = \frac{1}{2}
\end{displaymath}
\begin{flushleft}
\item[iii)]
\begin{description}
Bedingte Wahrscheinlichkeit
\end{description}
\end{flushleft}
\begin{displaymath}
	\frac{ \mathbb{P} \{(2,3)\}}{\mathbb{P} \{(1,4),(2,3),(4,1),(3,2)\}} = \frac{1}{4}
\end{displaymath}

\section*{3. Aufgabe}
Eine Urne enthält b rote und c schwarze Kugeln. Eine Kugel wird zufällig gezogen, ihre
Farbe notiert und mit weiteren d Kugeln gleicher Farbe in die Urne zurückgelegt.
\begin{enumerate}[i)]
\item
Bedingte Wahrscheinlichkeit
$$
\text{1. Zug: rot} = \frac{b}{b+c} \text{ schwarz} = \frac{c}{b+c}
$$
$$
\text{2. Zug: }\frac{c}{b+c} * \frac{c+d}{b+c+d} + \frac{b}{b+c} * \frac{c}{b+c+d}
$$
\item
\end{enumerate}
$$
\frac{\frac{c}{b+c} * \frac{c+d}{b+c+d}}{\frac{c}{b+c} * \frac{c+d}{b+c+d} + \frac{b}{b+c} * \frac{c}{b+c+d}}
$$

\section*{4. Aufgabe}
Vier Maschinen sind hintereinander geschaltet. Die erste funktioniert mit Wahrschein-
lichkeit p, jede weitere mit Wahrscheinlichkeit p oder p=2, je nach dem, ob die vorherige
Maschine funktioniert hat. Wenn mindestens drei der Maschinen funktioniert haben, ist
das Ergebnis korrekt. Wie groß ist die Wahrscheinlichkeit dafur?
\\[1cm]
\begin{displaymath}
p_1 = p 
\end{displaymath}
\begin{displaymath}
p_2 = p * p_1 + (1-p_1)* \frac{p}{2} 
\end{displaymath}
\begin{displaymath}
p_3 = p * p_2 + (1-p_2)* \frac{p}{2} 
\end{displaymath}
\begin{displaymath}
p_4 = p * p_3 + (1-p_3)* \frac{p}{2} 
\end{displaymath}
\\[0.6cm]
\begin{align}
\Rightarrow \mathbb{P} (\text{Mindestens \medspace 3 \medspace funktionieren}) &=  p_1 * p_2 * p_3 * p_4 + (1-p_1)*p_2*P_3*p_4 + p_1*(1-p_2)*P_3*p_4 + \nonumber \\
 &\qquad {} p_1*p_2*(1-p_3)*p_4 + p_1*p_2*p_3(1-p_4) \nonumber
\end{align}
\end{document}
