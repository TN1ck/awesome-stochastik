\documentclass[10pt,a4paper,parskip=half]{scrartcl}
\usepackage[utf8]{inputenc}
\usepackage{amsmath}
\usepackage{amsfonts}
\usepackage{amssymb}
\usepackage{mathpazo}
\usepackage{tikz}
\usetikzlibrary{patterns}
\usepackage{stmaryrd} % Für den Widerspruchsblitz :D
\usepackage[left=1cm, right=1cm,
top=1cm, bottom=1cm]{geometry}
\usepackage{fullpage}
\usepackage[german]{babel}
\usepackage{enumerate}
\setlength{\unitlength}{1cm}
\newcommand{\N}{\mathbb{N}}
\newcommand{\PP}{\mathbb{P}}
\newcommand{\A}{\mathcal{A}}
\newcommand{\R}{\mathbb{R}}
\parindent 0mm

\usepackage{color}
\usepackage{enumerate}



\begin{document}
\begin{center}
\textsc{\Large{Stochastik für Informatiker - Hausaufgabe 1}} \\
\end{center}
\begin{tabbing}
Tom Nick \hspace{1.4cm}\= 342225\\
Alexander Mühle\> 339497\\
Maximilian Bachl\> 341455
\end{tabbing}
\section*{1. Aufgabe}
	\begin{enumerate}
		\item[a)] 
			\begin{align*}
				\Omega &:= \{(w_1,w_2) \mid w_i \in \{1,2,3,4,5,6\}, i \in \{1,2\}\} \\
				 \PP(x) &= \frac{1}{6^2} = \frac{1}{36}
			\end{align*}
		\item[b)]
			\begin{enumerate}
				\item[i)] \begin{align*} 
									i &= \{(w_1,w_2) \in \Omega \mid w_1 + w_2 \le 5 \}\\
									\PP(i) &= \frac{10}{36}
							\end{align*}
				\item[ii)] \begin{align*} 
									ii &= \{(w_1,w_2) \in \Omega \mid w_1\text{ mod } 2 = 1 \land w_2\text{ mod } 2 = 1 \}\\
									\PP(i) &= \frac{1}{4}
							\end{align*}
				\item[iii)] \begin{align*} 
									iii &= \{(w_1,w_2) \in \Omega \mid (w_1 + w_2) \text{ mod } 2 = 1\}\\
									\PP(i) &= \frac{1}{2}
							\end{align*}
				\item[iv)] \begin{align*} 
									i &= \{(w_1,w_2) \in \Omega \mid ( w_1 * w_2) \text{ mod } 2 = 0 \}\\
									\PP(i) &= \frac{3}{4}
							\end{align*}
			\end{enumerate}
	\end{enumerate}
\section*{2. Aufgabe}
\begin{enumerate}[\quad(i)]
	\item $$ A_i :=  \Omega \setminus (A_1 \cup A_2 \cup A_3)$$
	\item $$ A_{ii} := \overline{A_i} $$
	\item $$ A_{iii} :=  (\overline{A_1} \cup \overline{A_2}) \cap (\overline{A_1} \cup \overline{A_3}) \cap (\overline{A_2} \cup \overline{A_3}) $$
	\item $$A_{iv} = A_{iii} \setminus  A_i $$
	\item $$ A_{v} = \Omega \setminus (A_1 \cap A_2 \cap A_3) $$
	\item $$ A_v \setminus A_{iii} $$
\end{enumerate}
\section*{3. Aufgabe}
	\begin{enumerate}
		\item[a)] Die Anzahl der \textit{verschiedenen} Paare einer Farbe beträgt $\sum^n_{k=1} k = \frac12 \cdot n \cdot (n+1)$. 
		Somit gibt es also $n \cdot (n+1)$ verschiedene Möglichkeiten zwei Kugeln gleicher Farbe zu ziehen,
		bei jeweils $n$ Kuglen von jeder Farbe.
		Die Anzahl allgemein möglicher Züge kann berechnet werden mit: $\binom n2$,
		 also beträgt die Wahrscheinlichkeit:
		 
		 \[ \frac{n(n+1)}{\binom {2n}2} = \frac{n(n+1)}{\frac{2n!}{2! \cdot (2n - 2)!}} = \frac{n+1}{2n -1} \]
	\item[b)] Wenn man nicht zwei gleich Kuglen zieht, zieht man 2 verschiedenfarbige, also ist die wahrscheinlichkeit dafür:
	$$1 - \frac{n+1}{2n - 1} $$	
	\end{enumerate}
\section*{4. Aufgabe}
\end{document}

