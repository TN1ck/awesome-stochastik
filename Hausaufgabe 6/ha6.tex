\documentclass[10pt,a4paper,parskip=half]{scrartcl}
\usepackage[utf8]{inputenc}
\usepackage{amsmath}
\usepackage{amsfonts}
\usepackage{amssymb}
\usepackage{mathpazo}
\usepackage{tikz}
\usetikzlibrary{patterns}
\usepackage[left=1cm, right=1cm,
top=1cm, bottom=1cm]{geometry}
\usepackage{fullpage}
\usepackage[german]{babel}
\usepackage{enumerate}
\setlength{\unitlength}{1cm}
\newcommand{\N}{\mathbb{N}}
\newcommand{\PP}{\mathbb{P}}
\newcommand{\A}{\mathcal{A}}
\newcommand{\R}{\mathbb{R}}
\parindent 0mm

\usepackage{color}
\usepackage{enumerate}



\begin{document}
\begin{center}
\textsc{\Large{Stochastik für Informatiker - Hausaufgabe 6}} \\
\end{center}
\begin{tabbing}
Tom Nick \hspace{1.4cm}\= 342225\\
Alexander Mühle\> 339497\\
Maximilian Bachl\> 341455
\end{tabbing}
\subsection*{Aufgabe 1}
X = Anzahl von Köpfen in den letzten drei Würfen

Y = Anzahl von Köpfen in den ersten und zweiten Würfen

\begin{enumerate}[(i)]
\item  
$ \Omega = \{ (w_1,w_2,w_3,w_3) | w_1,w_2,w_3,w_4 \in \{K,Z\} \} $ \\
$ \forall \omega \in \Omega . \mathbb{P}\{\omega\} = \frac{1}{2}^4 = \frac{1}{16}  $

\item
Verteilung von X:\\
$X(\Omega) = \{0,1,2,3\}$\\
$\mathbb{P}(X = 0) = \frac{2}{16} = \frac{1}{8}$\\
$\mathbb{P}(X = 1) = \frac{6}{16} = \frac{3}{8}$\\
$\mathbb{P}(X = 2) = \frac{6}{16} = \frac{3}{8}$\\
$\mathbb{P}(X = 3) = \frac{2}{16} = \frac{1}{8}$\\

Verteilung von Y:\\
$Y(\Omega) = \{0,1,2\}$\\
$\mathbb{P}(Y = 0) = \frac{4}{16} = \frac{2}{8}$\\
$\mathbb{P}(Y = 1) = \frac{8}{16} = \frac{4}{8}$\\
$\mathbb{P}(Y = 2) = \frac{4}{16} = \frac{2}{8}$\\

\item\leavevmode\vadjust{\vspace{-\baselineskip}}\newline
Verteilung von X und Y
\begin{tabular}{c | c c c | c}
$X/Y$ & $0$ & $1$ & $2$ & $\mathbb{P}(X=x) $\\ \hline
0 & $\frac{1}{16}$ & $\frac{1}{16}$ & $0$ & $\frac{1}{8}$ \\
1 & $\frac{1}{8}$ & $\frac{3}{16}$ & $\frac{1}{16}$ & $\frac{3}{8}$ \\
2 & $\frac{1}{16}$ & $\frac{3}{16}$ & $\frac{1}{8}$ & $\frac{3}{8}$ \\
3 & $0$ & $\frac{1}{16}$ & $\frac{1}{16}$ & $\frac{1}{8}$ \\ \hline
$\mathbb{P}(Y=y)$ & $\frac{2}{8}$ & $\frac{4}{8}$ & $\frac{2}{8}$ & 1
\end{tabular}

\item
Nein! Da: \\\begin{center}
 $\mathbb{P}(X  = 3, Y = 0) = 0 \neq \mathbb{P}(X = 3) \cdot \mathbb{P}(Y = 0) = \frac{1}{8} \cdot \frac{2}{8} = \frac{2}{64} = \frac{1}{32}$
\end{center}
\end{enumerate}


\subsection*{Aufgabe 2}

\subsection*{Aufgabe 3}

\subsection*{Aufgabe 4}

\end{document}