\documentclass[10pt,a4paper,parskip=half]{scrartcl}
\usepackage[utf8]{inputenc}
\usepackage{amsmath}
\usepackage{amsfonts}
\usepackage{amssymb}
\usepackage{mathpazo}
\usepackage{booktabs}
\usepackage{tikz}
\usetikzlibrary{patterns}
\usepackage[left=1cm, right=1cm,
top=1cm, bottom=1cm]{geometry}
\usepackage{fullpage}
\usepackage[german]{babel}
\usepackage{enumerate}
\setlength{\unitlength}{1cm}
\newcommand{\N}{\mathbb{N}}
\newcommand{\PP}{\mathbb{P}}
\newcommand{\A}{\mathcal{A}}
\newcommand{\R}{\mathbb{R}}
\parindent 0mm

\usepackage{color}
\usepackage{enumerate}
\renewcommand*\arraystretch{1.5}


\begin{document}
\begin{center}
\textsc{\Large{Stochastik für Informatiker - Hausaufgabe 7}} \\
\end{center}
\begin{tabbing}
Tom Nick \hspace{1.4cm}\= 342225\\
Alexander Mühle\> 339497\\
Maximilian Bachl\> 341455
\end{tabbing}
\section*{Aufgabe 1}
\begin{enumerate}[(i)]
\item
Die Wahrscheinlichkeit, dass wir die ersten $n-1$ mal nicht Schwarz ziehen und am Ende eine Schwarze:
$$\mathbb P(X = n) = \left(\prod_{i=1}^{n-1} 1-\frac{M}{M+N}\right) \frac{M}{M+N}$$
\item
Wir ziehen die Wahrscheinlichkeiten aus der vorigen Aufgabe bis exklusive $k$ von 1 ab, weil der Text \textit{mindestens} besagt.
$$ \mathbb P(X \ge k) = 1 - \sum_{i=1}^{k-1} P(X = i)$$
\end{enumerate}
\end{document}