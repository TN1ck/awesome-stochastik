\documentclass[10pt,a4paper,parskip=half]{scrartcl}
\usepackage[utf8]{inputenc}
\usepackage{amsmath}
\usepackage{amsfonts}
\usepackage{amssymb}
\usepackage{mathpazo}
\usepackage{booktabs}
\usepackage{tikz}
\usetikzlibrary{patterns}
\usepackage[left=1cm, right=1cm,
top=1cm, bottom=1cm]{geometry}
\usepackage{fullpage}
\usepackage[german]{babel}
\usepackage{enumerate}
\setlength{\unitlength}{1cm}
\newcommand{\N}{\mathbb{N}}
\newcommand{\PP}{\mathbb{P}}
\newcommand{\A}{\mathcal{A}}
\newcommand{\R}{\mathbb{R}}
\parindent 0mm

\usepackage{color}
\usepackage{enumerate}
\renewcommand*\arraystretch{1.5}


\begin{document}
\begin{center}
\textsc{\Large{Stochastik für Informatiker - Hausaufgabe 7}} \\
\end{center}
\begin{tabbing}
Tom Nick \hspace{1.4cm}\= 342225\\
Alexander Mühle\> 339497\\
Maximilian Bachl\> 341455
\end{tabbing}
\section*{Aufgabe 1}
\begin{enumerate}[(i)]
\item
Die Wahrscheinlichkeit, dass wir die ersten $n-1$, wobei $n \in \{0\dots N\}$ mal nicht Schwarz ziehen und am Ende eine Schwarze:
$$\mathbb P(X = n) = \left(\prod_{i=1}^{n-1} 1-\frac{M}{M+N}\right) \frac{M}{M+N} = \left(1- \frac{M}{M+N}\right)^{n-1}\frac{M}{M+N}$$
Das Ganze ist eine andere Schreibweise der geom.~Verteilung.
\item
Wir ziehen die Wahrscheinlichkeiten aus der vorigen Aufgabe bis exklusive $k$, wobei $k \in \{0\dots N\}$ von 1 ab, weil der Text \textit{mindestens} besagt.
$$ \mathbb P(X \ge k) = 1 - \sum_{i=1}^{k-1} P(X = i)$$
\end{enumerate}
\section*{Aufgabe 2}
\begin{enumerate}[(i)]
\item
$X$ ist die Anzahl an Stochastik-Büchern unter 4 gezogenen Büchern.
$$\mathbb P(X=4) = \frac{{13\choose 4} {12 \choose 0}}{{25 \choose 4}} = \frac{13}{230}$$
\item
$Y$ ist die Anzahl an Analysis-Büchern unter 4 gezogenen Büchern.
$$\mathbb P(Y=4) = \frac{{12\choose 4} {13 \choose 0}}{{25 \choose 4}} = \frac{9}{230}$$
\item
\begin{align*}
\mathbb P(X\ge 1) &= 1 - \mathbb P(X= 0)\\
&= 1 - \mathbb P(Y= 4)\\
&= 1 - \frac{13}{230}\\
&= 1 - \frac 9 {230}\\
&= \frac{221}{230}
\end{align*}
\end{enumerate}
\section*{Aufgabe 3}
$X$ ist die Anzahl an unbrauchbaren Birnen in einer Lieferung.

Exakt mithilfe der Binomialverteilung.
\begin{align*}
\mathbb P(X \ge 2) &= 1 - \mathbb P(X = 0) - \mathbb P(X = 1)\\
&= 1 - {500 \choose 0 } 0.001^0 0.999^500 - {500 \choose 1 } 0.001^1 0.999^499\\
&\approx 0.09
\end{align*}
Und nun mithilfe der Poisson-Annäherung:

$\lambda = 500 \cdot 0.001$
\begin{align*}
\mathbb P(X \ge 2) &= 1 - \mathbb P(X = 0) - \mathbb P(X = 1)\\
&= 1 - \frac{0.5^0}{0!}e^{-0.5} - \frac{0.5^1}{1!}e^{-0.5}\\
& \approx 0.09
\end{align*}
\section*{Aufgabe 4}
Aus der Aufgabenstellung wissen wir:
$$\mathbb P(X=k) = \frac{\lambda^k}{k!}\cdot e^{-\lambda}$$
$$\mathbb P(Y=k) = \frac{\mu^k}{k!}\cdot e^{-\mu}$$
Dann wissen wir auch:
$$P(X+Y=n) = \sum^n_{i=0} \frac{\lambda^i}{i!}\cdot e^{-\lambda} \cdot\frac{\mu^{n-i}}{(n-i)!}\cdot e^{-\mu}$$
Da ja nach Aufgabenstellung $X+Y=n$ und $n$ fest ist folgt aus der Definition der bedingten Wahrscheinlichkeit:
\begin{align*}P(X=k|X+Y=n) &= \frac{\frac{\lambda^k}{k!}\cdot e^{-\lambda} \cdot\frac{\mu^{n-k}}{(n-k)!} \cdot e^{-\mu}}{\sum^n_{i=0} \frac{\lambda^i}{i!}\cdot e^{-\lambda} \cdot\frac{\mu^{n-i}}{(n-i)!}\cdot e^{-\mu}}\\
&=\frac{\frac{\lambda^k}{k!} \cdot\frac{\mu^{n-k}}{(n-k)!}}{\sum^n_{i=0} \frac{\lambda^i}{i!} \cdot\frac{\mu^{n-i}}{(n-i)!}}\\
&=\frac{\frac{\lambda^k\mu^{n-k}}{k!(n-k)!}}{\sum^n_{i=0} \frac{\lambda^i\mu^{n-i}}{i!(n-i)!}}\\
&=\frac{n!\frac{\lambda^k\mu^{n-k}}{k!(n-k)!}}{\sum^n_{i=0} n!\frac{\lambda^i\mu^{n-i}}{i!(n-i)!}}\\
&=\frac{{n\choose k} \lambda^k\mu^{n-k}}{\sum^n_{i=0} {n\choose i} \lambda^i\mu^{n-i}}\\
\intertext{mithilfe des binom.~Lehrsatzes}\\
&=\frac{{n\choose k} \lambda^k\mu^{n-k}}{(\lambda + \mu)^n}\\
&={n\choose k}\left(\frac{\lambda}{\lambda + \mu}\right)^k\left(\frac{\mu}{\lambda + \mu}\right)^{n-k}\\
\end{align*}
Somit sehen wir, dass die Binomialverteilung die Parameter $n=n$, wobei $n$ ja fest und gegeben ist, und $p=\frac \lambda {\lambda+\mu}$ hat.
\end{document}




