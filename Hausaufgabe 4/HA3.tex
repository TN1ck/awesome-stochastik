\documentclass[10pt,a4paper,parskip=half]{scrartcl}
\usepackage[utf8]{inputenc}
\usepackage{amsmath}
\usepackage{amsfonts}
\usepackage{amssymb}
\usepackage{mathpazo}
\usepackage{tikz}
\usetikzlibrary{patterns}
\usepackage[left=1cm, right=1cm,
top=1cm, bottom=1cm]{geometry}
\usepackage{fullpage}
\usepackage[german]{babel}
\usepackage{enumerate}
\setlength{\unitlength}{1cm}
\newcommand{\N}{\mathbb{N}}
\newcommand{\PP}{\mathbb{P}}
\newcommand{\A}{\mathcal{A}}
\newcommand{\R}{\mathbb{R}}
\parindent 0mm

\usepackage{color}
\usepackage{enumerate}



\begin{document}
\begin{center}
\textsc{\Large{Stochastik für Informatiker - Hausaufgabe 4}} \\
\end{center}
\begin{tabbing}
Tom Nick \hspace{1.4cm}\= 342225\\
Alexander Mühle\> 339497\\
Maximilian Bachl\> 341455
\end{tabbing}
\section*{Aufgabe 1}
\textbf{Karten} $:= \{2,3,4,5,6,7,8,9,10,B,D,K,A\}$ \\
$\Omega := \{C_x, P_x, K_y, H_z \mid x \in \textbf{Karten}, y \in \textbf{Karten}\setminus\{K\}, z \in (\textbf{Karten}\setminus\{K\})\cup\{K1,K2\}\}$ \\
$\forall x \in \Omega. \PP(x) = \frac{1}{52}$ \\
\textbf{Caro Karten} $:= \{C_x \mid x \in \textbf{Karten}\}$ \\
\textbf{Alle roten Karten} $:= \textbf{Caro Karten}   \cup \{H_x \mid x \in (\textbf{Karten}\setminus\{K\})\cup\{K1,K2\}\}$ \\
\begin{enumerate}
	\item $A$ und $B$ sind abhängig: \\
			 \begin{align*}(A \cap B) &= \PP(\{H_{K1}, H_{K2}\}) = \frac{3}{52} \neq \PP(A) \cdot \PP(B) \\
			 &=  \PP(\{H_{K1},H_{K2},C_{K},P_K\}) \cdot \PP(\textbf{Alle roten Karten}) = \frac{4}{52} \cdot \frac{27}{52} = \frac{27}{676} 			 
			 \end{align*}
	\item $B$ und $C$ sind abhängig: \\
			 \begin{align*}(B \cap C) &= \PP(B) = \PP(\textbf{Caro Karten} ) \\
			 &= \frac{13}{52} \neq \PP(A) \cdot \PP(B) \\
			 &=   \PP(\textbf{Caro Karten}) \cdot \PP(\textbf{Alle roten Karten}) = \frac{13}{52} \cdot \frac{27}{52} 
			 \end{align*}
	\item $A$ und $C$ sind unabhängig: \\
			 \begin{align*}\PP(A \cap C) &= \PP(\{C_K\} ) \\
			 &= \frac{1}{52} = \PP(A) \cdot \PP(B) = \PP(\{H_{K1},H_{K2},C_{K},P_K\}) \cdot \PP(\textbf{Caro Karten}) = \frac{4}{52} \cdot \frac14 = \frac{1}{52}
			 \end{align*}
\end{enumerate}
Da gezeigt wurde, dass $A$ und $B$ abhängig sind, $B$ und $C$ auch, aber $A$ und $C$ nicht, folgert aus der Abhängigkeit von $A$, $B$ und $B$,$C$ nicht die Abhängikeit von $A$ und $C$.
\section*{Aufgabe 2}
$\Omega := \{(w_1,w_2) \mid w_i \in \{1,2,3,4,5,6\}, i \in \{1,2\}\}$ \\
 $\forall x \in \Omega.\PP(x) = \frac{1}{6^2} = \frac{1}{36}$
\begin{enumerate}
	\item $A_i$ und $B_i$ abhängig?
		\begin{itemize}
			\item $A_1$ und $B_1$: \\
				\begin{align*}
					\PP(A_1 \cap B_1) &= \PP(\{(1,3),(1,4),(1,5),(1,6)\})  = \frac{4}{36} \\
					&= \PP(A_1) \cdot \PP(B_1) =  \PP(\{(1,x) \mid x \in \{1..6\}\}) \cdot \PP(\{(x,y) \mid x,y \in \{1..6\}, y \ge 3\}) \\
					&= \frac{1}{6} \cdot \frac{4}{6} = \frac{4}{36} \\
					&\Rightarrow \textbf{unabhängig!}
				\end{align*}
			\item $A_2$ und $B_2$: \\
				\begin{align*}
					\PP(A_2 \cap B_2) &= \PP(\{(5,5),(6,5)\})  = \frac{2}{36} \\
					&\neq \PP(A_2) \cdot \PP(B_2) \\
					&= \PP(\{(x,y) \mid x,y \in \{1..6\}, x+y \ge 10\}) \cdot \PP(\{(x,5) \mid x \in \{1..6\}\}) \\
					&= \frac{1}{6} \cdot \frac{1}{6} = \frac{1}{36} \\
					&\Rightarrow \textbf{abhängig!}
				\end{align*}
			\item $A_3$ und $B_3$: \\
				\begin{align*}
					\PP(A_3 \cap B_3) &= \PP(\{(1, 2), (2, 1), (2, 3), (3, 2), (4, 1), (4, 3), (5, 2),(6, 1),(6, 3)\})  = \frac{9}{36} \\
					&= \PP(A_3) \cdot \PP(B_3) \\
					& = \PP(\{(x,y) \mid x,y \in \{1..6\}, (x + y) \text{ mod } 2 = 1\}) \cdot \PP(\{ (x,y) \mid x,y \in \{1..6\}, y \le 3\}) \\
					&= \frac{1}{2} \cdot \frac{1}{2} = \frac{1}{4} \\
					&\Rightarrow \textbf{unabhängig!}
				\end{align*}
		\end{itemize}
		\item 
		Annahme: $A$ und $B$ sind unabhängig, d.h. $\PP(A \cap B) = \PP(A) \cdot \PP(B)$
		\begin{align*}
			\PP(A^C \cap B) &= \PP(B \setminus (A \cap B)) \\
			&= \PP(B) - \PP(A \cap B) \\
			&= \PP(B) - \PP(A) \cdot \PP(B) \\
			&= \PP(B)(1 - \PP(A)) \\
			&= \PP(B)\cdot\PP(A^C) \\
			\PP(A^C \cap B^C) &= \PP(A^C \setminus (A^C \cap B)) \\
			&= \PP(A^C) - \PP(A^C \cap B) \\
			&= \PP(A^C) - \PP(B)\cdot\PP(A^C)  \\
			&= \PP(A^C)\cdot(1 - \PP(B)) \\
			&= \PP(A^C)\cdot\PP(B^C)
		\end{align*}
\end{enumerate}
\section*{Aufgabe 3}
$O_1 := $ Dokument ist im ersten Ordner.
$DO_1^C :=$ Das Dokument wurde nicht im ersten Ordner gefunden (beim ersten durchwühlen).
$\frac{1}{x} :=$ Die Wahrscheinlichkeit, dass es in einem der Ordner ist.
\begin{align*}
	\PP(DO_1^C) &= \frac{2}{x} + \frac{1}{x}(1 - p_1) \\
	\PP(O_1) &= \frac{1}{x} \\
	\PP(DO_1^C \mid O_1) &= 1 - p_i \\
	\PP(O_1 \mid DO_1^C) &= \frac{\PP(O_1 \cap DO_1^C)}{\PP(DO_1^C)} \\	
	&= \frac{\PP(DO_1^C \mid O_1) \cdot \PP(O_1)}{\PP(DO_1^C)} \\
	&= \frac{(1-p_1) \cdot \frac{1}{x}}{\frac{2}{x} + \frac{1}{x} \cdot (1-p_1)} \\
	&= \frac{1 - p_1}{x - p_1}
\end{align*}
\section*{Aufgabe 4}
Es gibt 4 Möglichkeiten, die Kugeln zu ziehen, nämlich: RRR, RSR, SSR und SRR, die im nachfolgenden dargestellt werden. Man muss nur darauf achten, dass man die Kugeln, nach jedem Zug in die andere Urne legt.
\begin{align*} 
\mathbb P(\text{letzte Kugel aus $U_1$ ist rot}) =&~ \frac{r_1}{r_1+s_1}\frac{r_2 + 1}{r_2+s_2 + 1}\frac{r_1}{r_1+s_1}\\
+&~ \frac{r_1}{r_1+s_1}\frac{s_2}{r_2+s_2 + 1}\frac{r_1-1}{r_1+s_1}\\
+&~ \frac{s_1}{r_1+s_1}\frac{s_2+1}{r_2+s_2 + 1}\frac{r_1-1}{r_1+s_1}\\
+&~ \frac{s_1}{r_1+s_1}\frac{r_2}{r_2+s_2 + 1}\frac{r_1}{r_1+s_1}
\end{align*}


\end{document}
