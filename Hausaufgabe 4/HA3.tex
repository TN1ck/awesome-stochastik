\documentclass[10pt,a4paper,parskip=half]{scrartcl}
\usepackage[utf8]{inputenc}
\usepackage{amsmath}
\usepackage{amsfonts}
\usepackage{amssymb}
\usepackage{mathpazo}
\usepackage{tikz}
\usetikzlibrary{patterns}
\usepackage[left=1cm, right=1cm,
top=1cm, bottom=1cm]{geometry}
\usepackage{fullpage}
\usepackage[german]{babel}
\usepackage{enumerate}
\setlength{\unitlength}{1cm}
\newcommand{\N}{\mathbb{N}}
\newcommand{\PP}{\mathbb{P}}
\newcommand{\A}{\mathcal{A}}
\newcommand{\R}{\mathbb{R}}
\parindent 0mm

\usepackage{color}
\usepackage{enumerate}



\begin{document}
\begin{center}
\textsc{\Large{Stochastik für Informatiker - Hausaufgabe 4}} \\
\end{center}
\begin{tabbing}
Tom Nick \hspace{1.4cm}\= 342225\\
Alexander Mühle\> 339497\\
Maximilian Bachl\> 341455
\end{tabbing}
\section*{Aufgabe 4}
Es gibt 4 Möglichkeiten, die Kugeln zu ziehen, nämlich: RRR, RSR, SSR und SRR, die im nachfolgenden dargestellt werden. Man muss nur darauf achten, dass man die Kugeln, nach jedem Zug in die andere Urne legt.
\begin{align*} 
\mathbb P(\text{letzte Kugel aus $U_1$ ist rot}) =&~ \frac{r_1}{r_1+s_1}\frac{r_2 + 1}{r_2+s_2 + 1}\frac{r_1}{r_1+s_1}\\
+&~ \frac{r_1}{r_1+s_1}\frac{s_2}{r_2+s_2 + 1}\frac{r_1-1}{r_1+s_1}\\
+&~ \frac{s_1}{r_1+s_1}\frac{s_2+1}{r_2+s_2 + 1}\frac{r_1-1}{r_1+s_1}\\
+&~ \frac{s_1}{r_1+s_1}\frac{r_2}{r_2+s_2 + 1}\frac{r_1}{r_1+s_1}
\end{align*}

\end{document}
